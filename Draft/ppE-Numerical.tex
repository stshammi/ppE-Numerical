%%%%%%%%%%%%%%%%%%%%%%%%%%%%%%%%%%%
%\documentclass[prd,aps,twocolumn,nofootinbib,showpacs,superscriptaddress]{revtex4-1}
\documentclass[prd,twocolumn,nofootinbib]{revtex4-1}
%\documentclass[prd,aps,nofootinbib,showpacs]{revtex4-1}
\usepackage{amsfonts}
\usepackage{amsmath}
\usepackage{amssymb}
\usepackage{bm}
\usepackage{dcolumn}
\usepackage[dvips]{graphicx}
\usepackage{graphics}
%\usepackage[latin1]{inputenc}
\usepackage{latexsym}
\usepackage{rotating}
\usepackage[colorlinks=true]{hyperref}
\usepackage{xspace} % Sensible space treatment at end of simple macros
\usepackage[usenames]{color}
\usepackage{mathrsfs}
\usepackage{multirow}
\usepackage{pifont}
\usepackage{enumitem}
\usepackage{color}
%\usepackage{ulem}
%\usepackage{url}
\usepackage{appendix}
\usepackage[utf8]{inputenc}
%\usepackage[toc,page]{appendix}
\usepackage{comment}
%% Try to control orphans, widows, and extra whitespace
\widowpenalty=1000
\clubpenalty=1000
\raggedbottom

\definecolor {darkgreen}{rgb}{0.2,0.7,0.2}
\definecolor{purple}{rgb}{0.5,0,0.5}

\newcommand\be{\begin{equation}}
\newcommand\ba{\begin{eqnarray}}
\newcommand\ee{\end{equation}}
\newcommand\ea{\end{eqnarray}}
\newcommand\bw{\begin{widetext}}
\newcommand\ew{\end{widetext}}
\newcommand{\lb}{\left(}
\newcommand{\rb}{\right)}

\newcommand{\EDGB}{{\mbox{\tiny EdGB}}}
\newcommand{\BD}{{\mbox{\tiny BD}}}
\newcommand{\NC}{{\mbox{\tiny NC}}}
\newcommand{\PPE}{{\mbox{\tiny ppE}}}
\newcommand{\KG}{{\mbox{\tiny kh}}}
\newcommand{\EA}{{\mbox{\tiny EA}}}
\newcommand{\ST}{{\mbox{\tiny ST}}}
\newcommand{\NS}{{\mbox{\tiny NS}}}
\newcommand{\DCS}{{\mbox{\tiny dCS}}}
\newcommand{\GR}{{\mbox{\tiny GR}}}
\newcommand{\Gdot}{{\mbox{\tiny $\dot G$}}}
\newcommand{\GW}{{\mbox{\tiny GW}}}
\newcommand{\ky}[1]{\textcolor{blue}{\it{\textbf{ky: #1}}} }
\newcommand{\st}[1]{\textcolor{cyan}{\it{\textbf{st: #1}}} }

%\bibliographystyle{apsrev}
\begin{document}
\title{Not decided yet}

\author{Sharaban Tahura}
\affiliation{Department of Physics, University of Virginia, Charlottesville, Virginia 22904, USA.}

\author{Kent Yagi}
\affiliation{Department of Physics, University of Virginia, Charlottesville, Virginia 22904, USA.}

\begin{abstract}
To be written later
\end{abstract}

\date{\today}




\maketitle


%%%%%%%%%%%%%%%%%%%%%%%%%%


\section{Introduction}

\section{PPE Formalism}\label{section:ppE}

\section{Data Analysis Formalism}
%Fisher Analysis
%%Assumptions
We are going to use Fisher analysis~\cite{Cutler:1994ys} to estimate the determination errors of the non-GR parameters in different theories. Fisher analysis is valid for GW events with sufficiently large signal-to-noise (SNR) ratios. We make the assumptions that the detector noise is Gaussian and stationary. Let us write the detector output as
\be
s(t)=n(t)+h(t)\,,
\ee
where n(t) is the noise and h(t) is the GW signal. Let us also define the inner product of two quantities $A$ and $B$ in the following way:
 \be
 \left(A|B\right)=4Re\int_0^\infty df \frac{\tilde{A}^*(f)\tilde{B}^(f)}{S_n \left( f \right)}\,,
 \ee
where $\tilde{A}(f)$ is the Fourier component of $A$ and an asterisk ($*$) superscript means the complex conjugate. $S_n\left(f\right)$ is the noise spectral density. With above definitions, the probability distribution of the noise can be written as
 \be
P\left(n=n_0(t)\right) \propto \text{exp}\left[-\left(n_0|n_0\right)\right]\,,
\ee
and the SNR for a given signal $h(t)$ can be defined as 
\be
\rho\left[h\right]\equiv\sqrt{\left(h|h\right)}\,.
\ee
With the assumptions of Gaussian and stationary noise, the posterior probability distribution of a binary parameter $\theta^a$ takes the following form~:
\be
P\left(\theta^a|s\right) \propto p^{(0)}\left(\theta^a\right) \text{exp}\left[-\frac{1}{2} \Gamma_{ab} \Delta\theta^a\Delta \theta^b \right]\,,
\ee
where $\Delta \theta^a=\hat{\theta}^a-\theta^a$, with $\hat{\theta}^a$ being the maximum likelihood value of the parameter $\theta^a$. $p^{(0)}\left(\theta^a\right)$ gives the probability distribution of the prior information. $\Gamma_{ab}$ is called the Fisher information matrix which is defined as
\be
\Gamma_{ab}=\left(\partial_a h|\partial_b h \right)\,,
\ee
where $\partial_b=\frac{\partial}{\partial \theta^b}$. One can estimate the root-mean-square of $\Delta \theta^a$ by taking the square root of the diagonal elements of the inverse Fisher matrix $\Sigma^{ab}$ : 
\be
\Sigma^{ab}=\left(\tilde{\Gamma}^{-1}\right)^{ab}=\langle\Delta\theta^a\Delta \theta^b\rangle\,,
\ee
where $\tilde{\Gamma}_{ab}$ is defined by
\be
p^{(0)}\left(\theta^a\right) \text{exp}\left[-\frac{1}{2} \Gamma_{ab} \Delta\theta^a\Delta \theta^b \right]=\text{exp}\left[-\frac{1}{2} \tilde{\Gamma}_{ab} \Delta\theta^a\Delta \theta^b \right]\,.
\ee

%%binary parameters
We choose the following parameters as our variables for the Fisher analysis~:
\bw
\ba
\theta^a\equiv \lb \ln{\mathcal{M}_z}, \ln{\eta}, \ln{D_L}, \ln{t_0}, \chi, \phi_0, \alpha, \delta, \psi, \theta_{\text{inc}}, \beta_{\PPE}, \alpha_{\PPE} \rb\,.
\ea
\ew
Here, $\mathcal{M}_z$ is the the redshifted chirp mass and $\chi$ is the effective spin parameter~\footnote{The effective spin parameter is defined as $\chi\equiv\lb m_1 \chi_1+m_2\chi_2\rb /\lb m_1+m_2\rb$, where $\chi_1$ and $\chi_2$ are the dimensionless spins of $m_1$ and $m_2$ respectively.}. $\alpha, \delta, \psi$, and  $\theta_{\text{inc}}$ are the right ascension, polarization, declination, and inclination angles respectively in the detector frame. All other parameters have the same meanings as in Sec.~\ref{section:ppE}. 
%Monte-Carlo
Instead of choosing a sky-averaged waveform, we perform a Monte Carlo simulation. Namely, we uniformly distribute the polarization angle between $0$ to $\pi$ and the coalescence phase between $0$ to $2\pi$. For $\mathcal{M}_z, \eta, D_L, \chi, \alpha, \delta$, and $\theta_{\text{inc}}$ we use the full posterior sample released by LIGO~\cite{ligo:sample}. 
%%prior on spin
We impose the prior information such that the effective spin is less than 1.
%%Detectors
We use the detector sensitivity of Advanced LIGO (aLIGO) 01 run~\cite{LIGOScientific:2018mvr} and we consider the two detectors at Hanford and Livingstone. For simplicity, we assume that the Livingstone noise spectrum is identical to that of the Hanford~\cite{Yunes:2009yz}. For Fisher integration, the minimum frequency is taken to be 20 Hz while the maximum frequency is same as the cutoff frequency above which the signal power is negligible~\cite{Ajith:2009bn}.


%The rule of compound probability
Now we are going to discuss how we achieve the probability distribution of a non-GR parameter from the output of Fisher analysis with Monte Carlo. We set the  fiducial value of any non-GR parameter to be zero for our Fisher analysis. We perform the following integration numerically to obtain the compound probability density function~\footnote{If the distribution of a random variable $y$ depends on a parameter $x$, and if $x$ is not fixed rather a random variable, the marginal distribution of $y$ is called mixture distribution or compound probability distribution and is given by $P\left(y\right)=\int P\left(y|x\right) P\left(x\right)dx$. The underlying distribution $P\left(x\right)$ is called mixing distribution or latent distribution.~\cite{2016arXiv160204060R}} of any parameter $\xi$~:
\be
\label{eq3:1}
P\lb\xi\rb=\int P\lb \xi|\sigma_{\xi}\rb P\lb \sigma_{\xi}\rb d\sigma_{\xi}\,,
\ee
where $P\lb\xi\rb$ is the marginal (unconditional) probability density function of $\xi$. $P\lb \xi|\sigma_{\xi}\rb$ is the conditional probability density function of $\xi$ given $\sigma_\xi$, where $\sigma_\xi$ is the 90\% CL estimation error of $\xi$ obtained from the Fisher analysis. $P\lb\sigma_\xi\rb$ is the probability distribution of $\sigma_\xi$ which is achieved by performing Fisher analyses for the entire posterior sample.
%We perform the integration in Eq.~\ref{eq3:1} numerically to obtain the unconditional probability density function $P\lb\xi\rb$.
\section{Example Theories}
\subsection{Massive Gravity}
\subsection{Scalar-Tensor Theories}
\subsection{Einstein-Dilaton-Gauss-Bonnet Theory}
\subsection{Varying- $G$ Theory}
\section{Conlusion}
\section{Appendix}
%phenomB vs PhenomD for propagation mechanism
%------------------------------------------------------- 
\bibliography{ppE}
\end{document}
