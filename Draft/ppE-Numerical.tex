%%%%%%%%%%%%%%%%%%%%%%%%%%%%%%%%%%%
%\documentclass[prd,aps,twocolumn,nofootinbib,showpacs,superscriptaddress]{revtex4-1}
\documentclass[prd,twocolumn,nofootinbib]{revtex4-1}
%\documentclass[prd,aps,nofootinbib,showpacs]{revtex4-1}
\usepackage{amsfonts}
\usepackage{amsmath}
\usepackage{amssymb}
\usepackage{bm}
\usepackage{dcolumn}
\usepackage[dvips]{graphicx}
\usepackage{graphics}
%\usepackage[latin1]{inputenc}
\usepackage{latexsym}
\usepackage{rotating}
\usepackage[colorlinks=true]{hyperref}
\usepackage{xspace} % Sensible space treatment at end of simple macros
\usepackage[usenames]{color}
\usepackage{mathrsfs}
\usepackage{multirow}
\usepackage{pifont}
\usepackage{enumitem}
\usepackage{color}
\usepackage{ulem}
%\usepackage{url}
\usepackage{appendix}
\usepackage[utf8]{inputenc}
%\usepackage[toc,page]{appendix}
\usepackage{comment}
%% Try to control orphans, widows, and extra whitespace
\widowpenalty=1000
\clubpenalty=1000
\raggedbottom

\definecolor {darkgreen}{rgb}{0.2,0.7,0.2}
\definecolor{purple}{rgb}{0.5,0,0.5}

\newcommand\be{\begin{equation}}
\newcommand\ba{\begin{eqnarray}}
\newcommand\ee{\end{equation}}
\newcommand\ea{\end{eqnarray}}
\newcommand\bw{\begin{widetext}}
\newcommand\ew{\end{widetext}}
\newcommand{\lb}{\left(}
\newcommand{\rb}{\right)}

\newcommand{\EDGB}{{\mbox{\tiny EdGB}}}
\newcommand{\BD}{{\mbox{\tiny BD}}}
\newcommand{\NC}{{\mbox{\tiny NC}}}
\newcommand{\PPE}{{\mbox{\tiny ppE}}}
\newcommand{\KG}{{\mbox{\tiny kh}}}
\newcommand{\EA}{{\mbox{\tiny EA}}}
\newcommand{\ST}{{\mbox{\tiny ST}}}
\newcommand{\NS}{{\mbox{\tiny NS}}}
\newcommand{\DCS}{{\mbox{\tiny dCS}}}
\newcommand{\GR}{{\mbox{\tiny GR}}}
\newcommand{\Gdot}{{\mbox{\tiny $\dot G$}}}
\newcommand{\GW}{{\mbox{\tiny GW}}}
\newcommand{\ky}[1]{\textcolor{blue}{\it{\textbf{ky: #1}}} }
\newcommand{\kent}[1]{\textcolor{magenta}{\textbf{#1}} }
\newcommand{\st}[1]{\textcolor{cyan}{\it{\textbf{st: #1}}} }

%\bibliographystyle{apsrev}
\begin{document}
\title{Not decided yet}

\author{Sharaban Tahura}
\affiliation{Department of Physics, University of Virginia, Charlottesville, Virginia 22904, USA.}

\author{Kent Yagi}
\affiliation{Department of Physics, University of Virginia, Charlottesville, Virginia 22904, USA.}

\author{Zack Carson}
\affiliation{Department of Physics, University of Virginia, Charlottesville, Virginia 22904, USA.}

\begin{abstract}
\ky{I added Zack in the author list. Somewhere in the paper, we need to show the comparison of the PPE bounds with PhenomB (that you computed) and PhenomD (that Zack computed) to justify the use of PhenomB for negative PN corrections.}

To be written later
\end{abstract}

\date{\today}




\maketitle


%%%%%%%%%%%%%%%%%%%%%%%%%%


\section{Introduction}

\section{PPE Formalism}\label{section:ppE}

\section{Data Analysis Formalism}
%Fisher Analysis
%%Assumptions
We \sout{are going to use} \kent{adopt a} Fisher analysis~\cite{Cutler:1994ys} to estimate the \sout{determination} \kent{statistical} errors of the non-GR parameters in \sout{different} \kent{various} theories. \sout{Fisher} \kent{Such an} analysis is valid for GW events with sufficiently large signal-to-noise (SNR) ratios. We make the assumptions that the detector noise is Gaussian and stationary. Let us write the detector output as
\be
s(t)=n(t)+h(t)\,,
\ee
where \sout{$n(t)$ is the noise and $h(t)$ is the GW signal} \kent{$n(t)$ and $h(t)$ are the noise and the GW signal respectively}. Let us also define the inner product of two quantities $A(t)$ and $B(t)$ \kent{as} \sout{in the following way:}
 \be
 \left(A|B\right)=4\Re\int_0^\infty df \frac{\tilde{A}^*(f)\tilde{B}(f)}{S_n \left( f \right)}\,.
 \ee
Here $\tilde{A}(f)$ is the Fourier component of $A$, an asterisk ($*$) superscript means the complex conjugate and $S_n\left(f\right)$ is the noise spectral density. With the above definitions, the probability distribution of the noise can be written as
 \be
P\left(n=n_0(t)\right) \propto \text{exp}\left[-\left(n_0|n_0\right)\right]\,,
\ee
and the SNR for a given signal $h(t)$ can be defined as \ky{I removed $[h]$ from $\rho$.}
\be
\rho \equiv\sqrt{\left(h|h\right)}\,.
\ee
\sout{With} \kent{Under} the assumptions of a Gaussian and stationary noise, the posterior probability distribution of binary parameters $\theta^a$ takes the following form: \ky{No need to add a space before a colon.}
\be
P\left(\theta^a|s\right) \propto p^{(0)}\left(\theta^a\right) \text{exp}\left[-\frac{1}{2} \Gamma_{ab} \Delta\theta^a\Delta \theta^b \right]\,,
\ee
where $\Delta \theta^a=\hat{\theta}^a-\theta^a$ with $\hat{\theta}^a$ being the maximum likelihood values of $\theta^a$. $p^{(0)}\left(\theta^a\right)$ gives the probability distribution of the prior information, \kent{which we take to be in a Gaussian form for simplicity}. $\Gamma_{ab}$ is called the Fisher information matrix which is defined as
\be
\Gamma_{ab}=\left(\partial_a h|\partial_b h \right)\,,
\ee
where $\partial_b\equiv \frac{\partial}{\partial \theta^b}$. One can estimate the root-mean-square of $\Delta \theta^a$ by taking the square root of the diagonal elements of the inverse Fisher matrix $\Sigma^{ab}$: 
\be
\Sigma^{ab}=\left(\tilde{\Gamma}^{-1}\right)^{ab}=\langle\Delta\theta^a\Delta \theta^b\rangle\,,
\ee
where $\tilde{\Gamma}_{ab}$ is defined by
\be
p^{(0)}\left(\theta^a\right) \text{exp}\left[-\frac{1}{2} \Gamma_{ab} \Delta\theta^a\Delta \theta^b \right]=\text{exp}\left[-\frac{1}{2} \tilde{\Gamma}_{ab} \Delta\theta^a\Delta \theta^b \right]\,.
\ee

%%binary parameters
We choose the following parameters as our variables for the Fisher analysis: \ky{Did you use $\ln t_0$ instead of $t_0$? (Just checking.)}
\bw
\ba
\theta^a\equiv \lb \ln{\mathcal{M}_z}, \ln{\eta}, \ln{D_L}, \ln{t_0}, \chi, \phi_0, \alpha, \delta, \psi, \theta_{\text{inc}}, \beta_{\PPE}, \alpha_{\PPE} \rb\,.
\ea
\ew
Here, $\mathcal{M}_z$ is the the redshifted chirp mass and $\chi$ is the effective spin parameter~\footnote{The effective spin parameter is defined as $\chi\equiv\lb m_1 \chi_1+m_2\chi_2\rb /\lb m_1+m_2\rb$, where \kent{$\chi_A$ with $A=(1,2)$} \sout{$\chi_1$ and $\chi_2$} is the dimensionless spin of \sout{$m_1$ and $m_2$ respectively} \kent{the $A$th body}.}. $\alpha, \delta, \psi$, and  $\theta_{\text{inc}}$ are the right ascension, \sout{polarization}, declination, \kent{polarization} and inclination angles respectively in the detector frame. All other parameters have the same meanings as in Sec.~\ref{section:ppE}. 
%Monte-Carlo
\sout{Instead of choosing a sky-averaged waveform,} We perform a Monte Carlo simulation \kent{by using each set of the posterior samples released by LIGO~\cite{ligo:sample} for $(\mathcal{M}_z, \eta, D_L, \chi, \alpha, \delta$, $\theta_{\text{inc}})$, while we randomly sample the polarization angle $\psi$ and the coalescence phase $\phi_0$ in $[0,\pi]$ and $[0,2\pi]$ respectively.}
\sout{Namely, we uniformly distribute the polarization angle between $0$ to $\pi$ and the coalescence phase between $0$ to $2\pi$. For $\mathcal{M}_z, \eta, D_L, \chi, \alpha, \delta$, and $\theta_{\text{inc}}$ we use the full posterior sample released by LIGO~\cite{ligo:sample}. }
%%prior on spin
We impose the prior information such that the effective spin is less than 1.

%%Detectors
\ky{I started a new paragraph here. Please keep in mind that each paragraph should start with a topic sentence that summarizes that paragraph. In the previous paragraph, your topic sentence is ``We choose the following parameters as our variables for the Fisher analysis.'' So the explanation of the detector noise should be explained in a separate paragraph.}
We use the detector sensitivity of Advanced LIGO (aLIGO) O1 run~\cite{LIGOScientific:2018mvr} and we consider the two detectors at Hanford and Livingstone. For simplicity, we assume that the Livingstone noise spectrum is identical to that of the Hanford~\cite{Yunes:2009yz}. For Fisher integration, the minimum frequency is taken to be 20 Hz while the maximum frequency is same as the cutoff frequency above which the signal power is negligible~\cite{Ajith:2009bn}.


%The rule of compound probability
Now we are going to discuss how we achieve the probability distribution of a non-GR parameter from the output of \kent{a} Fisher analysis with a Monte Carlo \kent{simulation}. We set the  fiducial value of any non-GR parameter to be zero for our \sout{Fisher} \ky{avoid repetition} analysis. We perform the following integration numerically to obtain the compound probability density function~\footnote{If the distribution of a random variable $y$ depends on a parameter $x$, and if $x$ is not fixed rather a random variable, the marginal distribution of $y$ is called mixture distribution or compound probability distribution and is given by $P\left(y\right)=\int P\left(y|x\right) P\left(x\right)dx$. The underlying distribution $P\left(x\right)$ is called mixing distribution or latent distribution.~\cite{2016arXiv160204060R}} of any parameter $\xi$~:
\be
\label{eq3:1}
P\lb\xi\rb=\int P\lb \xi|\sigma_{\xi}\rb P\lb \sigma_{\xi}\rb d\sigma_{\xi}\,,
\ee
where $P\lb\xi\rb$ is the marginal (unconditional) probability density function of $\xi$. \kent{$P\lb \xi|\sigma_{\xi}\rb \propto \exp[-(\xi-\bar \xi)^2/2\sigma_{\xi}^2]$} is the conditional probability density function of $\xi$ \sout{given $\sigma_\xi$} \kent{which we assume to be a Gaussian distribution with a mean $\bar \xi$ and a standard deviation $\sigma_\xi$.} \sout{where $\sigma_\xi$ is the 90\% CL estimation error of $\xi$ obtained from the Fisher analysis.} $P\lb\sigma_\xi\rb$ is the probability distribution of $\sigma_\xi$ \sout{which is achieved by performing Fisher analyses} \kent{computed from the Fisher analysis} for the entire posterior sample.
%We perform the integration in Eq.~\ref{eq3:1} numerically to obtain the unconditional probability density function $P\lb\xi\rb$.
\section{Example Theories}
\subsection{Massive Gravity}
\subsection{Scalar-Tensor Theories}
\subsection{Einstein-Dilaton-Gauss-Bonnet Theory}
\subsection{Varying- $G$ Theory}
\section{Conlusion}
\section{Appendix}
%phenomB vs PhenomD for propagation mechanism
%------------------------------------------------------- 
\bibliography{ppE}
\end{document}
